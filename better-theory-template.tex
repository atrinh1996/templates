\documentclass[11pt]{article}
\usepackage[utf8]{inputenc} %% I added for characters

%%%%%% Given %%%%%%%%%%%%%%%%%%%%%%%%%%%%%%%%%%%%%%%%%%%%%%%%%%%%%%%%%%%
\usepackage{alltt}
\usepackage{verbatim}
\usepackage{hyperref}
\usepackage{amsmath}
\usepackage{qtree}
\usepackage{semantic}
\usepackage{mathpartir}
%%%%%%%%%%%%%%%%%%%%%%%%%%%%%%%%%%%%%%%%%%%%%%%%%%%%%%%%%%%%%%%%%%%%%%%%

%%%%%%%%%%%%%%%%%%%%%%%%%%%%%%%%%%%%%%%%%%%%%%%%%%%%%%%%%%%%%%%%%%%%%%%%
\usepackage{enumitem} % listing style (bullet lists)
\usepackage{amssymb, latexsym} % label=\alph* for enumerate
%%%%%%%%%%%%%%%%%%%%%%%%%%%%%%%%%%%%%%%%%%%%%%%%%%%%%%%%%%%%%%%%%%%%%%%%

%%% GIVEN for Calculational Proof %%%%%%%%%%%%%%%%%%%%%%%%%%%%%%%%%%%%%%
\usepackage{array}
\usepackage{graphicx}
% \usepackage{amsmath}
\usepackage{times}
\renewcommand\ttdefault{aett}

\newcommand\pterm[1]{\multicolumn{2}{l}{#1}\\}
\newcommand\pfeqonly[1]{%
   \mskip 15mu{}={} & \{\mbox{#1}\} \\
}
\newcommand\pfeq[2][]{\pfeqonly{#1}\pterm{#2}}
\newcommand\ptermmbox[1]{\multicolumn{2}{l}{\mbox{#1}}\\}
\newenvironment{codeproof}{\[\advance\extrarowheight by 2pt 
                             \let\pterm=\ptermmbox\begin{array}{c@{}l}}
                     {\end{array}\]\ifhmode\unskip\par\fi\csname @endparenv\endcsname}

\let\Tt=\ttfamily
\let\nwendquote=\relax
%%%%%%%%%%%%%%%%%%%%%%%%%%%%%%%%%%%%%%%%%%%%%%%%%%%%%%%%%%%%%%%%%%%%%%%%


%%% GIVEN %%%%%%%%%%%%%%%%%%%%%%%%%%%%%%%%%%%%%%%%%%%%%%%%%%%%%%%%%%%%%%
\newcommand{\br}[1]{\langle #1 \rangle}
\newcommand{\state}[4]{\langle {#1,#2,#3,#4} \rangle}
\newcommand{\evalr}[2][{}]{\state{#2}{\xi#1}{\phi}{\rho#1}}
%%%%%%%%%%%%%%%%%%%%%%%%%%%%%%%%%%%%%%%%%%%%%%%%%%%%%%%%%%%%%%%%%%%%%%%%

%%%%%%%%%%%%%%%%%%%%%%%%%%%%%%%%%%%%%%%%%%%%%%%%%%%%%%%%%%%%%%%%%%%%%%%%
\newlength{\toppush}
\setlength{\toppush}{2\headheight}
\addtolength{\toppush}{\headsep}

% Page head Styling
% Name and subject of the class
\def\subjnum{Comp 105}          % Class Number
\def\subjname{Programming Languages}       % Class Name

% Name of the student, university name and which semester
\def\doheading#1#2#3{\vfill\eject\vspace*{-\toppush}%
  \vbox{\hbox to\textwidth{{\bf} \subjnum: \subjname \hfil Amy Bui}%
    \hbox to\textwidth{{\bf} Homework 2 \hfil#3\strut}%
    \hrule}}

%Command for the title of the document (Homework 0)
\newcommand{\htitle}[1]{\vspace*{1.25ex plus 1ex minus 0ex}%
\begin{center}
    {\large\bf #1}
\end{center}} 
%%%%%%%%%%%%%%%%%%%%%%%%%%%%%%%%%%%%%%%%%%%%%%%%%%%%%%%%%%%%%%%%%%%%%%%%

%%%%%%%%%%%%%%%%%%%%%%%%%%%%%%%%%%%%%%%%%%%%%%%%%%%%%%%%%%%%%%%%%%%%%%%%
% use \lstinline! xxx ! or \begin{lstlisting} ... \end{lstlisting}
\usepackage{listings}

\usepackage{color}
\definecolor{light-gray}{gray}{0.97} % shade of grey
\definecolor{dkgreen}{rgb}{0,0.6,0}
\definecolor{gray}{rgb}{0.5,0.5,0.5}
\definecolor{mauve}{rgb}{0.58,0,0.82}

% \begin{lstlisting}[...] ... \end{lstlisting}
\lstset{frame=none,
    language=C++,
    aboveskip=3mm,
    belowskip=3mm,
    stepnumber=1, % set to 0 if you don't like line nums
    showstringspaces=false,
    columns=flexible,
    basicstyle={\small\ttfamily},
    numbers=left,
    numberstyle=\color{black},
    keywordstyle=\color{blue},
    commentstyle=\color{dkgreen},
    stringstyle=\color{mauve},
    backgroundcolor=\color{light-gray},
    breaklines=true,
    breakatwhitespace=false,
    tabsize=2
}
%%%%%%%%%%%%%%%%%%%%%%%%%%%%%%%%%%%%%%%%%%%%%%%%%%%%%%%%%%%%%%%%%%%%%%%%

%%%%%%%%%%%%%%%%%%%%%%%%%%%%%%%%%%%%%%%%%%%%%%%%%%%%%%%%%%%%%%%%%%%%%%%%
\usepackage{xcolor}
%% https://tex.stackexchange.com/questions/401750/quick-and-short-command-for-coloring-one-word
\newcommand\shorthandon{\catcode`@=\active \catcode`^=\active \catcode`*=\active }
\newcommand\shorthandoff{\catcode`@=12 \catcode`^=7 \catcode`*=12 }
\shorthandon
\def@#1@{\textcolor{red}{#1}}%
\def^#1^{\textcolor{blue}{#1}}%
\def*#1{\string#1}
\shorthandoff
%% useage: \textcolor{red}{text here}
%%%%%%%%%%%%%%%%%%%%%%%%%%%%%%%%%%%%%%%%%%%%%%%%%%%%%%%%%%%%%%%%%%%%%%%%

\begin{document}

\doheading{2}{title}{abui02}


%%%%% Given %%%%%%%%%%%%%%%%%%%%%%%%%%%%%%%%%%%%%%%%%%%%%%%%%%%%%%%%%%%%%
% \parskip=0.8\baselineskip plus 2pt
% \parindent=0pt



%%%%%%%%%%%%%%%%%%%%%%%%%%%%%%%%%%%%%%%%%%%%%%%%%%%%%%%%%%%%%%%%%%%%%%%%
\section*{Calculational proof}
    Here are some examples of \LaTeX commands that could help you write a
    calculational proof.

    Writing a code chunk on a line: \texttt{(length (append xs ys))}.

    And here's a chunk of a calculational proof:
    \begin{codeproof}
    \pterm{{\Tt{}(length (append xs ys))\nwendquote}}
    \pfeqonly{ by assumption, $\mathtt{xs} = \mbox{\texttt{(cons z zs)}}$ }
    \pterm{\Tt{}(length (append (cons z zs) ys))}
    \pfeqonly{ append-cons law }
    \pterm{\Tt{}(length (cons z (append zs ys))}
    \end{codeproof}


%%%%%%%%%%%%%%%%%%%%%%%%%% End of Section A %%%%%%%%%%%%%%%%%%%%%%%%%%%%




%%%%%%%%%%%%%%%%%%%%%%%%%% Rules %%%%%%%%%%%%%%%%%%%%%%%%%%%%

\section*{Inference Rules}
% Here are some examples of LaTeX commands that could help you on the opsem
% assignment.

Verbatim code: \verb|(if x 2 3)|

Another way to put text in code font: \texttt{x}.

Putting text in the font to name rules: \textsc{IfFalse}.

% Here is an example of two rules appearing one over the other:

\begin{mathpar}
\inferrule*[Right=\textsc{Literal}]
          {\ }
          {\evalr{\textsc{literal}(v)}
               \Downarrow
           \state{v}{\xi}{\phi}{\rho}}
\end{mathpar}
\begin{mathpar}
\inferrule*[Right=\textsc{FormalVar}]
          {x \in \texttt{dom}\ \rho}
           {\evalr{\textsc{var}(x)}
                \Downarrow
            \state{\rho(x)}{\xi}{\phi}{\rho}}
\end{mathpar}

% And here is an example of a derivation (where I've shortened the abstract syntax
\textsc{literal} to \textsc{lit}):

\begin{mathpar}
\inferrule*[Left=\textsc{FormalAssign}]
  {\texttt{n} \in \texttt{dom}\ \rho
   \and
   \inferrule*[Right=\textsc{Literal}]
     {\ }
     {\evalr{\textsc{lit}(\texttt{0})} \Downarrow \evalr{0}} 
  }
  {\evalr{\textsc{set}(\textsc{var}(\texttt{n}), 
                      \textsc{lit}(\texttt{0}))}
   \Downarrow 
   \state{0}{\xi}{\phi}{\rho\{\texttt{n} \mapsto 0\}}
  }
\end{mathpar}
%%%%%%%%%%%%%%%%%%%%%%%%%%%%%%%%%%%%%%%%%%%%%%%%%%%%%%%%%%%%%%%%%%%%%%%%

%%% WhileEnd Rule
\begin{mathpar}
    \inferrule*[Right=\textsc{(WhileEnd)}]
    {
      \evalr{e_1}
        \Downarrow
      \state{v_1}{\xi'}{\phi}{\rho'}
      \quad v_1 = 0
    }
    {
      \evalr{\textsc{while}(e_1, e_2)}{}
        \Downarrow
      \state{0}{\xi'}{\phi}{\rho'}
    }
\end{mathpar}




% %%% FormalVar rule
\begin{mathpar}
  \inferrule*[Right=\textsc{(FormalVar)}]
  {
    x \in \texttt{dom}\ \rho
  }
  {
    \evalr{\textsc{var}(x)} \Downarrow \state{\rho(x)}{\xi}{\phi}{\rho}
  }
\end{mathpar}

% %%% FormalAssign Rule
\begin{mathpar}
    \inferrule*[Right=\textsc{(FormalAssign)}]
    {
      x \in \texttt{dom}\ \rho
        \qquad
      \evalr{e} \Downarrow \state{v}{\xi'}{\phi}{\rho'}
    }
    {
      \evalr{\textsc{set}(x, e)} \Downarrow \state{v}{\xi'}{\phi}{\rho'\{x \mapsto v\}}
    }
\end{mathpar}


%%% GlobalVar rule
\begin{mathpar}
    \inferrule*[Right=(\textsc{GlobalVar})]
    {
      x \notin \textsf{dom}\ \rho
      \qquad
      x \in \textsf{dom}\ \xi
    }
    {
      \evalr{\textsc{var}(x)}
      \Downarrow
      \state{\xi(x)}{\xi}{\phi}{\rho}
    }
\end{mathpar}

%%% GlobalAssign rule
\begin{mathpar}
  \inferrule*[Right=(\textsc{GlobalAssign})]
  {
    x \notin \textsf{dom}\ \rho
      \qquad
    x \in \textsf{dom}\ \xi
      \qquad
    \evalr{e} \Downarrow \state{v}{\xi'}{\phi}{\rho'}
  }
  {
    \evalr{\textsc{set}(x, e)}
    \Downarrow
    \state{v}{\xi'\{x \mapsto v\}}{\phi}{\rho'}
  }
\end{mathpar}


%%% If True rule
\begin{mathpar}
    \inferrule*[Right=(\textsc{IfTrue})]
    {
        \evalr{e_1} \Downarrow \state{v_1}{\xi'}{\phi}{\rho'}
        \and
        v_1 \neq 0
        \and
        \state{e_2}{\xi'}{\phi}{\rho'} \Downarrow \state{v_2}{\xi''}{\phi}{\rho''}
    }
    {
        \evalr{\textsc{if}(e_1, e_2, e_3)} \Downarrow \state{v_2}{\xi''}{\phi}{\rho''}
    }
\end{mathpar}


%%% If False rule
\begin{mathpar}
  \inferrule*[Right=(\textsc{IfFalse})]
  {
      \evalr{e_1} \Downarrow \state{v_1}{\xi'}{\phi}{\rho'}
      \and
      v_1 = 0
      \and
      \state{e_3}{\xi'}{\phi}{\rho'} \Downarrow \state{v_3}{\xi''}{\phi}{\rho''}
  }
  {
      \evalr{\textsc{if}(e_1, e_2, e_3)} \Downarrow \state{v_3}{\xi''}{\phi}{\rho''}
  }
\end{mathpar}
% \pagebreak

%%%%%%%%%%%%%%%%%%%%%%%% End of Inference Rules%%%%%%%%%%%%%%%%%%%%%%%%%



%%%%%%%%%%%%%%%%%%%%%%%%%%%%%%%%%%%%%%%%%%%%%%%%%%%%%%%%%%%%%%%%%%%%%%%%
\section*{Part A: Talking Operational Semantics (20\%)}



%%%%%%%%%%%%%%%%%%%%%%%%%% End of Section A %%%%%%%%%%%%%%%%%%%%%%%%%%%%


%%%%%%%%%%%%%%%%%%%%%%%%%%%%%%%%%%%%%%%%%%%%%%%%%%%%%%%%%%%%%%%%%%%%%%%%
\section*{Part B: Operational Semantics and Language Design (35\%)}




%%%%%%%%%%%%%%%%%%%%%%%%%% End of Section B %%%%%%%%%%%%%%%%%%%%%%%%%%%%


%%%%%%%%%%%%%%%%%%%%%%%%%%%%%%%%%%%%%%%%%%%%%%%%%%%%%%%%%%%%%%%%%%%%%%%%
\section*{Part C: Derivatives, Proofs, and Metatheory (45\%)}



%%%%%%%%%%%%%%%%%%%%%%%%%% End of Section C %%%%%%%%%%%%%%%%%%%%%%%%%%%%


\end{document}
