\documentclass[11pt]{article}


\usepackage{alltt}
\usepackage{verbatim}
\usepackage{hyperref}
\usepackage{amsmath}
\usepackage{qtree}
\usepackage{semantic}
\usepackage{mathpartir}

\newcommand{\br}[1]{\langle #1 \rangle}
\newcommand{\state}[4]{\langle {#1,#2,#3,#4} \rangle}
\newcommand{\evalr}[2][{}]{\state{#2}{\xi#1}{\phi}{\rho#1}}




\begin{document}

\parskip=0.8\baselineskip plus 2pt
\parindent=0pt

Here are some examples of LaTeX commands that could help you on the opsem
assignment.

Verbatim code: \verb|(if x 2 3)|

Another way to put text in code font: \texttt{x}.

Putting text in the font to name rules: \textsc{IfFalse}.

Here is an example of two rules appearing one over the other:

\begin{mathpar}
\inferrule*[Right=\textsc{Literal}]
          {\ }
          {\evalr{\textsc{literal}(v)}
               \Downarrow
           \state{v}{\xi}{\phi}{\rho}}
\end{mathpar}
\begin{mathpar}
\inferrule*[Right=\textsc{FormalVar}]
          {x \in dom \rho}
           {\evalr{\textsc{var}(x)}
                \Downarrow
            \state{\rho(x)}{\xi}{\phi}{\rho}}
\end{mathpar}

And here is an example of a derivation (where I've shortened the abstract syntax
\textsc{literal} to \textsc{lit}):

\begin{mathpar}
\inferrule*[Left=\textsc{FormalAssign}]
  {\texttt{n} \in dom \rho
   \and
   \inferrule*[Right=\textsc{Literal}]
     {\ }
     {\evalr{\textsc{lit}(\texttt{0})} \Downarrow \evalr{0}} 
  }
  {\evalr{\textsc{set}(\textsc{var}(\texttt{n}), 
                      \textsc{lit}(\texttt{0}))}
   \Downarrow 
   \state{0}{\xi}{\phi}{\rho\{\texttt{n} \mapsto 0\}}
  }
\end{mathpar}




\end{document}
